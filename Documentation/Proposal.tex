\documentclass[11pt]{extarticle}

\usepackage{cite}
\usepackage{array}
\usepackage{tabularx}

\usepackage{booktabs}% More proffesional look of tables.
\usepackage{siunitx}% An awesome package for typesetting and manipulation numbers and units.
\usepackage{caption}% Better control over caption
\usepackage{lipsum}% Example text

\usepackage[left=2cm, right=3cm, top=2cm]{geometry}

\title{Smart Waste Bin}
\author{A Project Proposal}
\date{January 2019}
\author{Kiranmai Arva}

\begin{document}

\maketitle
\section*{Introduction}

This paper describes a modification of the simple project Smart Bin. In the existing project, when an object is taken near to the dustbin it will automatically open the lid of the dustbin this is done using Ultrasonic Sensor and Servo Motor, instead of the user opening the lid manually. This project is referred from the Arduino project hub. Few enhancements are made on the proposed project based on the current model.
\section*{Novel Contribution}
This project aims to replicate the original project with few enhancements. In the existing project when an object is brought near to the bin it will open the lid of the dustbin. Some enhancements to be done with regards to the original project is when dustbin is full a Buzzer will ring, and LED will blink. At the same time, a notification is sent to the user’s mobile. In this project, both Buzzer and LED can be controlled using the smartphone instead button. Apart from above the user is also notified with the garbage pick off days so that user can place their bin outside. With respect to this project the ultrasonic sensor senses the data from the real world, that is it detects the object when it is nearby, and it also senses the amount of the level the bin is filled. With the computer code written and uploaded into the microcontroller of Arduino, it can control the Buzzer, LED and send notifications to the smartphone.

\section*{Motivation}
Garbage releases toxic gases which causes air pollution. Garbage disposal is very important to society, it not only causes air pollution but also contaminates water and leads to several health issues, so garbage management is vital for the general wellbeing of the public. Technology improved drastically that nowadays in each and every aspect technology is applied. The traditional method of garbage disposal is no more a part of society as it requires a lot of human power. Automated machinery has come to lift bins and dispose of the garbage. The concept of smart bin came into existence however smart the system is people forget to dispose of their dustbins properly in the home. They constantly need provocations or alarms when the bin is filled and get acknowledged with the garbage pick off days so that they can keep their bins outside. All the remainders are notified and can be controlled using smartphones as today's everything is smartphone mechanized.

\section*{Materials Required}
\begin{itemize}
  
   \item •	Arduino UNO
    \item •	Ultrasonic ping sensor
    \item 	•	Dustbin
    \item •	Servo meter
    \item•	Buzzer
   \item •	LED
   \item •	Smart phone
\item•	Bluetooth Module
\item•	GSM Module

\end{itemize}

\section*{Milestones}
Although the goals are ambitious the Following are the significant milestones planned to be achieved.
\begin{table}[h]                           
 \centering
    \begin{tabular}{|c|c|l|}
    \hline
    

     Milestone 1  & March 13  &All parts gathered together  \\ \hline
     Milestone 2  & March 17 & Assembling and building the hardware \\ \hline
     Milestone 2  & March 24 & Works on code part and Alarm  \\ \hline
     Milestone 2  & March 31 & Works on Software part of smart phone to receive notifications  \\ \hline
     Milestone 2  & March 18 & Notifies the user with message, stop the buzzer using smart phone \\ \hline
    

    \end{tabular}
   
\end{table}





Even though I don’t have enough experience with hardware with great effort and hard work, it is anticipated that all the milestones are achieved timely as mentioned above. Hope without any issues I will try to complete my final milestone successfully. 

\section*{Team Roles}
  All the roles like Gathering the components, Building the Hardware, Programming, Documentations are done by me.
  
 \section*{Summary}
 Many people does not remember to empty their bins unless stinky smell comes out of the bin, so a remainder alert is sent to smart phones, as soon as the bin is full it will notify the user with a buzz sound and an LED. Both LED and Buzzer is controlled by smartphone instead button. Apart from this user is also acquainted with garbage pick off days as per the schedule so that they can keep their bins outside.
 
 
 \begin{thebibliography}{1}

\bibitem @Article{ArduinoProjectHub.com,title={How to make SMART DUSTBIN},url={https://create.arduino.cc/projecthub/aakash11/how-to-make-smart-dustbin-7340cb}
 }



\end{thebibliography}







\end{document}
